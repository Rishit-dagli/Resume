\documentclass{resume} % Use the custom resume.cls style

\usepackage[left=0.4 in,top=0.4in,right=0.4 in,bottom=0.4in]{geometry}
\usepackage{xcolor}
\usepackage{natbib}
\bibliographystyle{unsrtnat}
\usepackage{hyperref}
\definecolor{fluorescentyellow}{rgb}{0.8, 1.0, 0.0}
\newcommand{\cfbox}[2]{%
    \colorlet{currentcolor}{.}%
    {\color{#1}%
    \fbox{\color{currentcolor}#2}}%
}
\renewcommand{\bibsection}{}
% Document margins
\newcommand{\tab}[1]{\hspace{.2667\textwidth}\rlap{#1}} 
\newcommand{\itab}[1]{\hspace{0em}\rlap{#1}}
\name{Rishit Dagli} % Your name
% You can merge both of these into a single line, if you do not have a website.
\address{"I draw a pleasure in thinking"} 
\address{\href{mailto:hello@rishit.tech}{hello@rishit.tech} \\ \href{https://github.com/Rishit-dagli}{github.com/Rishit-dagli} \\ \href{https://www.rishit.tech}{rishit.tech} \\ \href{https://www.linkedin.com/in/rishit-dagli/}{linkedin.com/in/rishit-dagli/}}  %

\begin{document}

\begin{rSection}{Education}

{\bf Bachelors of Science in Computer Science (1st year)}, University of Toronto \hfill {2022-2026}\\
Received a scholarship from Geoffrey Hinton and Vector Institute.\\
Received an entry scholarship from the University of Toronto.\\
Research assistant under \href{https://davidlindell.com/}{Prof. David Lindell}\\
Part of the \href{https://www.dgp.toronto.edu/}{DGP Lab} (for AI and Computer Vision) and \href{https://www.cs.toronto.edu/theory/}{Theory Research Group}\\
Part of the \href{http://cssu.ca/}{CS Student Union} Council.

{\bf High School}, Narayana Junior College \hfill {2020-2022}\\
Received scholarship by Narayana\\
Top 1\% Nationally in Junior Science Olympiad\\
Made past International Math Olympiad qualifier rounds

{\bf Summer School}, Stanford \hfill {2019}\\
Passed with distinction for the course "Statistical Learning", a course on supervised learning.
\end{rSection}

\begin{rSection}{SKILLS}
Python • TensorFlow • Machine Learning • Kubernetes • GCP • Git • Linux • C++ • HTML • CSS • Firebase\\
Android • Dart • Kotlin • SQL
\end{rSection}

\begin{rSection}{PUBLICATIONS}
A few of my recent publications are:
\nocite{*}
\bibliography{references}
\end{rSection}

\begin{rSection}{HONORS AND AWARDS}
The most recent ones are:

{\bf Invited to give a TEDx and TED-Ed talk} \hfill {2020-2021}\\
{\bf Research Grant by Google}, sponsors costs working on the CPPE-5 paper \hfill {2021-2022}\\
{\bf Google Open Source Expert prize}, given to individuals doing great work and contributing \\ to the open-source AI ecosystem \hfill {2022}\\
{\bf Civo and Steve Wozniak Scholarship}, also gives an opportunity to have breakfast with Steve Wozniak \hfill {2022}\\
{\bf Linux Foundation Scholarship}, with sponsored trips to KubeCon \hfill {2022}\\
{\bf TensorFlow Community Spotlight}, received the award for my recent Machine Learning paper,\\awarded to ML projects making using TensorFlow and of high-quality \hfill {2022}\\
{\bf OpenJS Foundation Scholarship}, with a sponsored trip to OpenJS World \hfill {2022}\\
{\bf PapersWithCode top contributor award}, received the award for my work in the field of ML \hfill {2022}\\
{\bf Research Grant by Intel}, sponsors costs working on the queuing paper \hfill {2021}\\
{\bf TensorFlow}, thanked publicly multiple times by TensorFlow Team for my open-source code\\contributions to TensorFlow on the TensorFlow GitHub releases page \hfill {2021-2022}\\
{\bf Software Grant by Google Cloud}, sponsors infrastructure for my open-source projects \hfill {2021}\\
{\bf PyTorch top-contributor}, being one of the top contributors to PyTorch, I was featured for an \\ interview with Meta on CNBC \hfill {2021}\\
{\bf Microsoft Green Hackathon}, Won the Microsoft Green Hackathon and was featured on\\Microsoft Blog and YouTube \hfill {2021}\\
{\bf freeCodeCamp top-contributor awards}, being one of the top contributors to freeCodeCamp,\\ the most popular open-source project \hfill {2021}\\
{\bf HackerNoon AI Genius Award}, given to individuals doing impactful work in the area of AI \hfill {2021}
\end{rSection}

\begin{rSection}{PROJECTS}
I also maintain and have built other open-source projects. \href{https://github.com/Rishit-dagli}{GitHub}

{\bf MIRNet-TFJS} \hfill {\href{https://github.com/Rishit-dagli/MIRNet-TFJS}{GitHub} • \href{https://tfhub.dev/rishit-dagli/mirnet-tfjs}{TF Hub}}
 \begin{itemize}
    \itemsep -3pt {} 
\item Featured multiple times on GitHub Trending.
\item This Implements a research paper and shows the TFJS model conversion and inference processes to allow it to run on the web.
\item This model is capable of enhancing low-light images up to a great extent.
\end{itemize}

{\bf Fast transformers} \hfill {\href{https://github.com/Rishit-dagli/Fast-Transformer}{GitHub}}
 \begin{itemize}
    \itemsep -3pt {} 
\item I implemented the paper "Fastformer: Additive Attention Can Be All You Need", a Transformer Variant that can handle far longer sequences than current ones.
\item Trended \# 1 on all of GitHub.
\end{itemize}

{\bf Gradient Centralization} \hfill {\href{https://github.com/RISHIT-DAGLI/GRADIENT-CENTRALIZATION-TENSORFLOW}{GitHub} • \href{https://pypi.org/project/gradient-centralization-tf/}{PyPI}}
 \begin{itemize}
    \itemsep -3pt {} 
\item I implemented the paper "Fastformer: Additive Attention Can Be All You Need", a Transformer Variant that can handle far longer sequences than current ones.
\item Trended \# 1 on all of GitHub.
\end{itemize}

{\bf TF Watcher} \hfill {\href{https://github.com/Rishit-dagli/TF-Watcher}{GitHub} • \href{https://www.tfwatcher.tech/}{Website} • \href{https://rishit-dagli.github.io/TF-Watcher/}{Docs}}
 \begin{itemize}
    \itemsep -3pt {} 
\item Industry scale project which allows monitoring Machine Learning training, evaluation, and prediction processes on mobile phones with as little as 2 lines of code and is compatible with all kinds of Machine Learning Code.
\item Won the Major League Hacking (MLH) Fellowship Hackathon.
\end{itemize}

{\bf Queuing Systems} \hfill {\href{https://link.springer.com/article/10.1007\%2Fs00500-021-05891-2}{Publication}}
 \begin{itemize}
    \itemsep -3pt {} 
\item Created a new Machine Learning recipe to deploy large video models on edge
devices and validated the research and received a research grant by Intel AI.
\item Used TensorFlow to implement algorithms and test them on multiple-edge hardware with OpenVino.
\end{itemize}

{\bf GLOM} \hfill {\href{https://github.com/Rishit-dagli/GLOM-TensorFlow}{GitHub}}
 \begin{itemize}
    \itemsep -3pt {} 
    \item The first-ever implementation of Geoffrey Hinton's idea paper, "How to represent part-whole hierarchies in a neural network"
    \item This allows advances made by several different groups of transformers, neural fields, contrastive representation learning, distillation, and capsules to be combined
\end{itemize}
\end{rSection}

\begin{rSection}{EXPERIENCE}

\textbf{Machine Learning Team} \hfill October 2022 - Ongoing\\
FINCH Mission \hfill \textit{Toronto, ON}
 \begin{itemize}
    \itemsep -3pt {} 
     \item On the Data Processing team for Field Imaging Nanosatellite for Crop residue Hyperspectral mapping (FINCH) mission.
     \item Working on using Machine Learning to make sense of the data returned by the satellite as well as on-orbit calibration, at the moment majorly working on image super-resolution and destriping.
     \item A 3U CubeSat mission projected to launch in Q4 2024 aboard a SpaceX Falcon 9 rocket.
 \end{itemize}
 
\textbf{Data Processing} \hfill 2021-2022\\
James Webb Space Telescope \hfill \textit{Mumbai, Remote}
 \begin{itemize}
    \itemsep -3pt {} 
     \item Worked on the data processing team for exoplanet detection on the James Webb Space Telescope.
     \item Developed Machine Learning algorithms as well as standard reciped to answer the question, "what can we infer from this image": is it an exoplanet, what kind of exoplanet, and what are its properties?
 \end{itemize}

 \textbf{Fellow} \hfill 2021 Summer\\
Major League Hacking (MLH) \hfill \textit{Mumbai, Remote}
 \begin{itemize}
    \itemsep -3pt {}
     \item MLH Fellowship is an internship alternative by MLH, Facebook, and GitHub for people who are excited about contributing to open-source.
     \item Developed and maintained with other fellows an end-to-end open-source industry-scale Machine Learning project "TF Watcher" following best practices.
    \item Won the End of Fellowship Hackathon.
 \end{itemize}
\end{rSection}

\begin{rSection}{LEADERSHIP AND COMMUNITY}
The major ones are:

{\bf Kubernetes 1.26 Release Team}, part of the team that release Kubernetes 1.26 \hfill {2022}\\
{\bf Kubeflow 1.17 Release Team}, part of the team that release Kubeflow 1.17 \hfill {2022}\\
{\bf GitHub Campus Experts} \hfill {2022}\\
• A student program by GitHub Education identifying leaders of technical student communities\\with extremely low selection rates\\
• Led the efforts for Field Day Canada, an unconference for technical communities\\
{\bf Blog}, impacted 100,000+ developers through my blogs \hfill {2019-2022}\\
• Authored multiple high-quality tutorials on TensorFlow Keras\\
• Authored multiple high-quality articles on freeCodeCamp\\
• Personal Blog\\
{\bf Technical Talks} \hfill {2019-2022}\\
Spoke at technical conferences sharing my knowledge with others. The major ones are:

\href{https://youtu.be/IeTibm880ys?list=LL&t=4318}{ML Research on Google's YouTube} • \href{https://kccncna2022.sched.com/event/184sa/cl-lightning-talk-open-source-kubernetes-and-cloudnative-from-the-eyes-of-a-high-schooler-rishit-dagli-narayana-junior-college-incoming-university-of-toronto}{KubeCon North America} • \href{https://youtu.be/LTtgaJLo378?t=7624}{TensorFlow Everywhere India} • \href{https://www.twitch.tv/videos/834577736}{Microsoft Global AI Student Conference} • \href{https://www.droidcon.com/media-detail?video=491094795}{Droidcon APAC} • \href{https://www.youtube.com/watch?v=5lr1IbTpwpI&feature=youtu.be&t=113}{Google Devfest} • \href{https://osslatam22.sched.com/event/15BrY}{Open Source Summit} • \href{https://www.google.com/url?q=https%3A%2F%2Fwww.twitch.tv%2Fvideos%2F1121993724%3Ft%3D01h36m00s&sa=D&sntz=1&usg=AOvVaw1pyXglg9cGjLBI_zwpsCrh}{Postman Summit}\\
{\bf Volunteering}\\
• Active open-source contributor with high-trust positions in TensorFlow, Kubernetes, Kubeflow,\\ freeCodeCamp, and more\\
• Active open-source maintainer and creator with multiple of my projects being on GitHub \#1 trending\\
• Core creator and maintainer of Microsoft's ML For Beginners, now with $>$40K GitHub Stars\\
• Helped underprivileged and orphan kids with 40 hours of practical education
about Computer Science.\\
{\bf TensorFlow Mumbai}\\
• Created a special interest group in my city for people interested in machine learning\\
• Mentor members for creating projects or solving queries related to ML\\
• Featured twice by Google Developers for my work as part of the community
\end{rSection}

\end{document}
